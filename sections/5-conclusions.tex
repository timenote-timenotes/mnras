\section{Conclusions} \label{sec:con}
In this work, we present the observational results for two \st{long-period} \ul{Long-period} comets, namely C/2019 L3 and C/2020 P3. 

An analysis of morphology was conducted on combined comet images with several enhancement techniques. Unfortunately, most of them show no obvious features, except for image of C/2019 L3 on \DTMdate{2021-5-14}, where there exists a small northeastward fan-shape structure under V and R filters. 

Circular aperture photometry was employed to measure the magnitudes of comets in various broadband filters within an aperture size of up to 1.5 times the full-width at half-maximum (FWHM) of the comet. 

We get the surface brightness profile for comet C/2019 L3 and calculate the gradient in the $0.5 \leqslant \lg{\rho} \leqslant 1.0$ range, finding that the average value is \num{-1.68}, suggesting a nonsteady coma. 

The $Af\rho$ values adjusted to phase angle of \ang{0} were derived based on photometric results in the aperture of \SI{e4}{\km}. Compared to other works, the $A(0)f\rho$ of C/2019 L3 is relatively high at $\thicksim${\SI{4}{\astronomicalunit}}, while that of C/2020 P3 is moderate at $\thicksim${\SI{7}{\astronomicalunit}}. The profiles of $Af\rho$ also show that both of the two comets possess a nonsteaty coma, since the $Af\rho$ values tend to decrease with aperture. 

Coma colors were described in the form of color indices and reddening. The $\mathrm{B} - \mathrm{V}$ colors of C/2019 L3 and C/2020 P3 are redder than the Sun, while the $\mathrm{V} - \mathrm{R}$ and $\mathrm{R} - \mathrm{I}$ colors of them are bluer than the Sun. The reddening of C/2019 L3 exhibits variations during the observational runs. 

