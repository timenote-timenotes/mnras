\section{Conclusions} \label{sec:con}
In this work, we present the observational results for two Long-period comets, namely C/2019 L3 and C/2020 P3. 

\st{An analysis of morphology was conducted on combined comet images with several enhancement techniques. Unfortunately, most of them show no obvious features, except for image of C/2019 L3 on \DTMdate{2021-5-14}, where there exists a small northeastward fan-shape structure under V and R filters. }
\ul{
    After applying the azimuthal renormalization method to the images of C/2019 L3 taken on \DTMdate{2021-5-14} by telescope Maksutov, a fan-shape structure can be observed in the northeast direction. This feature is visible in both V and R filters. 
}

\st{Circular aperture photometry was employed to measure the magnitudes of comets in various broadband filters within an aperture size of up to 1.5 times the full-width at half-maximum (FWHM) of the comet. }

We get the surface brightness profile for comet C/2019 L3 and calculate the gradient in the $0.5 \leqslant \lg{\rho} \leqslant 1.0$ range, finding that the average value is \num{-1.68}, suggesting a nonsteady coma. 

The $Af\rho$ values adjusted to phase angle of \ang{0} were derived based on photometric results in the aperture of \SI{e4}{\km}, 
\ul{
    where the R-band $A(0)f\rho$ values of C/2019 L3 range from {\SI{5043 +- 244}{\cm}} to {\SI{13611 +- 1874}{\cm}}, and those of C/2020 P3 range from {\SI{606 +- 31}{\cm}} to {\SI{869 +- 20}{\cm}}. 
}
Compared to other works, the $A(0)f\rho$ of C/2019 L3 is relatively high at $\thicksim${\SI{4}{\astronomicalunit}}, while that of C/2020 P3 is moderate at $\thicksim${\SI{7}{\astronomicalunit}}. The profiles of $Af\rho$ also show that both of the two comets possess a nonsteaty coma, since the $Af\rho$ values tend to decrease with aperture. 

Coma colors were described in the form of color indices and reddening. 
\ul{
    The average colors for C/2019 L3 are  
$\langle \mathrm{B} - \mathrm{V} \rangle = \num{0.75 +- 0.06}$, 
$\langle \mathrm{V} - \mathrm{R} \rangle = \num{0.27 +- 0.05}$, and 
$\langle \mathrm{R} - \mathrm{I} \rangle = \num{0.22 +- 0.05}$,  
while the average colors for C/2020 P3 are 
$\langle \mathrm{B} - \mathrm{V} \rangle = \num{0.95 +- 0.07}$, 
$\langle \mathrm{V} - \mathrm{R} \rangle = \num{0.29 +- 0.05}$, and 
$\langle \mathrm{R} - \mathrm{I} \rangle = \num{0.21 +- 0.05}$. 
}
The $\mathrm{B} - \mathrm{V}$ colors of C/2019 L3 and C/2020 P3 are redder than the Sun, while the $\mathrm{V} - \mathrm{R}$ and $\mathrm{R} - \mathrm{I}$ colors of them are bluer than the Sun. The reddening of C/2019 L3 exhibits variations during the observational runs, 
\ul{
    from {\SI{13.75 +- 1.07}{\percent/\kilo\angstrom}} to {\SI{-15.69 +- 0.37}{\percent/\kilo\angstrom}} with an average value of {\SI{0.94 +- 0.23}{\percent/\kilo\angstrom}}. 
}

