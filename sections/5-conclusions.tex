\section{Conclusions} \label{sec:con}
In \st{this work} \ul{summary}, we present the observational results \st{for two Long-period comets, namely} \ul{of comets} C/2019 L3 and C/2020 P3. \ul{The conclusions are as follows: }

After applying the azimuthal renormalization method to the images of C/2019 L3 taken on \DTMdate{2021-5-14} by telescope Maksutov, a fan-shape structure can be observed in the northeast direction. This feature is visible in both V and R filters. 

We get the surface brightness profile for comet C/2019 L3 and calculate the gradient in the $0.5 \leqslant \lg{\rho} \leqslant 1.0$ range, finding that the average value is \num{-1.68}, suggesting a nonsteady coma. 

The $Af\rho$ values adjusted to phase angle of \ang{0} were derived based on photometric results in the aperture of \SI{e4}{\km}, where the R-band $A(0)f\rho$ values of C/2019 L3 range from {\qty{5043 +- 244}{\cm}} to {\qty{13611 +- 1874}{\cm}}, and those of C/2020 P3 range from {\qty{606 +- 31}{\cm}} to {\qty{869 +- 20}{\cm}}.
Compared to other works, the $A(0)f\rho$ of C/2019 L3 is relatively high at $\thicksim${\qty{4}{\astronomicalunit}}, while that of C/2020 P3 is moderate at $\thicksim${\SI{7}{\astronomicalunit}}. The R-band $A(0)f\rho$ values of C/2019 L3 exhibit to decrease slightly as the heliocentric distance decreases during the obsevational runs, suggesting an abnormal activity. The profiles of $Af\rho$ also show that both of the two comets possess a nonsteaty coma, since the $Af\rho$ values tend to decrease with aperture. 

Coma colors were described in the form of color indices and reddening. 
The average colors for C/2019 L3 are  
$\langle \mathrm{B} - \mathrm{V} \rangle = \num{0.75 +- 0.06}$, 
$\langle \mathrm{V} - \mathrm{R} \rangle = \num{0.27 +- 0.05}$, and 
$\langle \mathrm{R} - \mathrm{I} \rangle = \num{0.22 +- 0.05}$,  
while the average colors for C/2020 P3 are 
$\langle \mathrm{B} - \mathrm{V} \rangle = \num{0.95 +- 0.07}$, 
$\langle \mathrm{V} - \mathrm{R} \rangle = \num{0.29 +- 0.05}$, and 
$\langle \mathrm{R} - \mathrm{I} \rangle = \num{0.21 +- 0.05}$. 
The $\mathrm{B} - \mathrm{V}$ colors of C/2019 L3 and C/2020 P3 are redder than the Sun, while the $\mathrm{V} - \mathrm{R}$ and $\mathrm{R} - \mathrm{I}$ colors of them are bluer than the Sun. Compared to other Long-period comets, both C/2019 L3 and C/2020 P3 exhibit distinct differences in their color indices. The reddening of C/2019 L3 calculated from B-band $Af\rho$ and R-band $Af\rho$ exhibits variations during the observational runs, from {\qty{13.75 +- 1.07}{\percent/\kilo\angstrom}} to {\qty{-15.69 +- 0.37}{\percent/\kilo\angstrom}} with an average value of {\qty{0.94 +- 0.23}{\percent/\kilo\angstrom}}. This could be attributed to  variations in the composition of the coma. As for C/2020 P3, the reddening could only be calculated for the date of \DTMdate{2021-5-12}, yielding a value of {\qty{-6.65 +- 0.01}{\percent/\kilo\angstrom}}. 

