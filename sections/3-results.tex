\section{Results} \label{sec:res}

\subsection{Morphology}

The original images of C/2019 L3 in the R band exhibit greater clarity compared to images in other bands, and they all resemble a star-like appearance. On the other hand, the original images of C/2020 P3 appear slightly faint, and the ones taken in the R band are also relatively clearer than those in other bands. Unlike C/2019 L3, some images of C/2020 P3 reveal the presence of a faintly extended tail. After applying the data reduction process mentioned before, we get one image for each filter on each observational date. Fig.~\ref{fig:combinedimg} is a thumbnail view of the reduction result with comet located in center of each small parts. Two different telescopes were involved in observing comet C/2019 L3 on \DTMdate{2021-5-14}, thus in Fig.~\ref{fig:combinedimg} there are two groups of thumbnails on this date with the upper one by telescope ZTSh and the lower one by telescope Maksutov, and the same applies to comet C/2020 P3 on \DTMdate{2021-5-12}. The black spots in several part of this view result from the psf subtraction process, as it is not always perfect for star subtraction. For comet C/2019 L3, the observational record is abundant. The I band filter was added in the observation from 2021 May, providing more data for this study. From the combined images of C/2019 L3, it is easily to notice that this comet is very round and seemingly isotropic. For comet C/2020 P3, the image resolution is lower and it is not easily discernible. Its extended coma resembling a tail is faintly visible, as seen in the images captured on \DTMdate{2021-5-12}. 


% \begin{figure}
%     \centering
%     \subcaptionbox{C/2019 L3 \added{(ATLAS)}}[\linewidth]{
%         \includegraphics[width=.9\linewidth]{combine1.pdf}
%         \includegraphics[width=.9\linewidth]{combine2.pdf} 
%         \includegraphics[width=.9\linewidth]{combine3.pdf}
%         \includegraphics[width=.9\linewidth]{combine4.pdf}
%     }
%     \caption{The processed images of comets C/2019 L3 \added{(ATLAS)} and C/2020 P3 \added{(ATLAS)} in different filters, with north at the top and east to the left. The field of view is $ 2^{\prime} \times 2^{\prime} $ for each thumbnail. Blue arrow shows the direction of the comets' velocity, and the orange arrow shows the direction to the Sun. }
%     \label{fig:combinedimg}
% \end{figure}

% \begin{figure}
%     \centering
%     \ContinuedFloat
%     \subcaptionbox{C/2019 L3 \added{(ATLAS)}}[\linewidth]{
%         \includegraphics[width=\linewidth]{combine5.pdf}
%         \includegraphics[width=\linewidth]{combine6.pdf} 
%         \includegraphics[width=\linewidth]{combine7.pdf} 
%     }
%     \caption{(Continued)}
% \end{figure}

% \begin{figure}
%     \centering
%     \ContinuedFloat
%     \subcaptionbox{C/2020 P3 \added{(ATLAS)}}[\linewidth]{
%         \includegraphics[width=\linewidth]{combine8.pdf}
%         \includegraphics[width=\linewidth]{combine9.pdf} 
%     }
%     \caption{(Continued)}
% \end{figure}

% \begin{figure}
%     \centering
%     \includegraphics[width=\linewidth]{azi_ren.pdf}
%     \caption{Azimuthal-renormalized image of C/2019 L3 \added{(ATLAS)} on \DTMdate{2021-5-14}, the field of view for each thunbnail is also $\ang{;2;}\times\ang{;2;}$. \label{fig:aziren}}
% \end{figure}
    
\begin{figure}
    \centering
    \subcaptionbox{C/2019 L3 (ATLAS), field of view: \qtyproduct{2x2}{\arcminute}}[\linewidth]{
        \includegraphics[width=.9\linewidth]{combine4-new.pdf}
        \includegraphics[width=.9\linewidth]{combine7-new.pdf}
    }
    \subcaptionbox{C/2020 P3 (ATLAS), field of view: \qtyproduct{0.8x0.8}{\arcminute}}[\linewidth]{
        \includegraphics[width=.9\linewidth]{combine9-new.pdf}
    }
    \caption{The processed images of comets C/2019 L3 (ATLAS) and C/2020 P3 (ATLAS) in different filters, with north at the top and east to the left. Blue arrow shows the direction of the comets' velocity, and the orange arrow shows the direction to the Sun. }
    \label{fig:combinedimg}
\end{figure}

% BVR
\begin{table}
    \centering
    \caption{Photometric results of comets C/2019 L3 (ATLAS) and C/2020 P3 (ATLAS) with apertures consistently set to a radius of \qty{40000}{\km}. }\label{tab:bvr}
    \begin{threeparttable}
        \resizebox{\linewidth}{!}{
        \begin{tabular}{cccccccc}
            \toprule
            Observation Time\tnote{1} & B & V & R & I & $\mathrm{B}-\mathrm{V}$ & $\mathrm{V}-\mathrm{R}$ & $\mathrm{R}-\mathrm{I}$\\
            \midrule
            \multicolumn{8}{l}{\textbf{C/2019 L3 (ATLAS)}} \\
            2021-03-28.729 &  \num{14.49 +- 0.09} & \num{13.88 +- 0.07} & \num{13.29 +- 0.15} & - & \num{0.62 +- 0.11} & \num{0.58 +- 0.16} & - \\
            2021-04-02.739 &  \num{14.56 +- 0.10} & \num{13.82 +- 0.04} & \num{13.34 +- 0.06} & - & \num{0.74 +- 0.11} & \num{0.47 +- 0.07} & - \\
            2021-04-03.719 &  \num{14.45 +- 0.05} & \num{13.77 +- 0.03} & \num{13.29 +- 0.07} & - & \num{0.68 +- 0.05} & \num{0.47 +- 0.08} & - \\
            2021-04-04.691 &  \num{14.46 +- 0.05} & \num{13.76 +- 0.02} & \num{13.52 +- 0.01} & - & \num{0.70 +- 0.05} & \num{0.24 +- 0.02} & - \\
            2021-04-08.691 &  \num{14.53 +- 0.05} & \num{13.67 +- 0.02} & \num{13.55 +- 0.01} & - & \num{0.86 +- 0.05} & \num{0.12 +- 0.03} & - \\
            2021-04-12.724 &  \num{14.62 +- 0.05} & \num{13.72 +- 0.01} & \num{13.56 +- 0.01} & - & \num{0.91 +- 0.05} & \num{0.16 +- 0.01} & - \\
            2021-04-14.702 &  \num{14.47 +- 0.04} & \num{13.76 +- 0.03} & \num{13.53 +- 0.02} & - & \num{0.72 +- 0.05} & \num{0.22 +- 0.03} & - \\
            2021-04-15.714 &  \num{14.45 +- 0.03} & \num{13.75 +- 0.01} & \num{13.51 +- 0.02} & - & \num{0.70 +- 0.04} & \num{0.24 +- 0.02} & - \\
            2021-05-04.736 &  \num{14.31 +- 0.04} & \num{13.74 +- 0.04} & \num{13.41 +- 0.04} & \num{13.24 +- 0.03} & \num{0.57 +- 0.05} & \num{0.33 +- 0.06} & \num{0.17 +- 0.05} \\
            2021-05-10.735 & \num{14.25 +- 0.03} & \num{13.51 +- 0.01} & \num{13.32 +- 0.03} & \num{13.08 +- 0.04} & \num{0.73 +- 0.04} & \num{0.19 +- 0.03} & \num{0.24 +- 0.05} \\
            2021-05-12.724 & \num{14.35 +- 0.03} & \num{13.48 +- 0.03} & \num{13.31 +- 0.02} & \num{13.11 +- 0.03} & \num{0.87 +- 0.04} & \num{0.17 +- 0.04} & \num{0.20 +- 0.04} \\
            2021-05-14.721 & \num{14.33 +- 0.04} & \num{13.43 +- 0.04} & \num{13.29 +- 0.02} & \num{13.06 +- 0.06} & \num{0.90 +- 0.06} & \num{0.14 +- 0.04} & \num{0.23 +- 0.06} \\
            \multicolumn{8}{l}{\textbf{C/2020 P3 (ATLAS)}} \\
            2021-04-02.775 & - & \num{18.10 +- 0.04} & \num{17.81 +- 0.03} & - & - & \num{0.29 +- 0.05} & - \\
            2021-05-11.744 & - & \num{18.23 +- 0.04} & \num{17.96 +- 0.03} & - & - & \num{0.27 +- 0.04} & - \\
            2021-05-12.749 & \num{18.97 +- 0.07} & \num{17.94 +- 0.06} & \num{17.71 +- 0.02} & \num{16.96 +- 0.03} & \num{1.04 +- 0.09} & \num{0.23 +- 0.06} & \num{0.76 +- 0.04} \\
            \bottomrule
        \end{tabular}
        }
        \begin{tablenotes}
            \item[1] UT time at the beginning of exposure
        \end{tablenotes}
    \end{threeparttable}
\end{table}

% \subsection{Nucleus size}

% For comets observed at large heliocentric distances that are commonly assumed to be inactive, their photometric R magnitude, denoted as $m_R$, can be utilized to derive the maximum estimate for the geometric cross-sectional area of the cometary nucleus. This estimation involves the use of the formula proposed by \cite{lamy_comet_2004} for asteroids observed at high phase angle that can be applied to a spherical body expressed as
% \begin{equation}
%     A_R a_N^2 < \num{2.24e22} r^2 \Delta^2 10^{0.4\left(m_{\odot} - m_R + \beta \alpha\right)}, 
% \end{equation}
% where $A_R = 0.04$ is the geometric albedo \citep{lamy_comet_2004}, $a_N$ the radius, $r$ the heliocentric distance in \si{\astronomicalunit}, $\Delta$ the geocentric distance in \si{\astronomicalunit}, $m_{\odot} = -27.15$ the R magnitude of the Sun \citep{willmer_absolute_2018}, $\alpha$ the phase angle, and $\beta = 0.04$ the phase coefficient \citep{lamy_comet_2004}. 
% Given that two comets in this work were active during the observational run, we set the photometry aperture equal to stellar FWHM and calculate the background value using a circular region near the comet nucleus to reduce the influence of cometary outbursts, then use differential photometry to get new R magnitude to provide a coarse estimation for the nucleus size. 
% The images of comet C/2019 L3 taken on \DTMdate{2021-5-4} and that of C/2020 P3 taken on \DTMdate{2021-5-12} exhibit high resolution and were captured at sufficiently distant heliocentric distance, from which we were able to estimate the upper limits of radii for two comets as follows: C/2019 L3 with a limit of {\SI{75.1 +- 3.5}{\km}}, and C/2020 P3 with a limit of {\SI{26.8 +- 0.7}{\km}}. 


\subsection{Surface brightness profile}

Based on the photometric results, the radial surface brightness profile (SBP) was computed as the function of the angular distance $\rho$ measured from the photocenter of comet. The measurement of surface brightness is largely affected by the accuracy of background subtraction. In order to obtain an accurate estimate of the background, we use the \texttt{photutils} package in Python and generate a 2D image of the background. 
In the case of a steady-state coma, the surface brightness $B$ is expected to follow a power-law relation with $\rho$ as $B \propto \rho^m$, where $-1.5 \leqslant m \leqslant -1.0$, and the index $m$ is often referred to as the gradient ($m=\dif\lg{B} \big/ \dif\lg{\rho}$). As the radiation pressure accelerates the dust particles, the value of $m$ decreases and approaches \num{-1.5} in the limiting case~\citep{jewitt_surface_1987}. Conversely, if the index $m$ falls below \num{-1.5}, it suggests the presence of nonsteady dust coma emission \citep{lowry_ccd_1999}.

Not only in single images does comet C/2019 L3 appear like a stellar, but also in stacked images. Nevertheless, the SBP of C/2019 L3 shows it clearly the excess flux in outer region compared with stellar SBP. In Fig.~\ref{fig:sbp} we report an example plot of the R-band SBP as a function of $\lg{\rho}$ for C/2019 L3 observed on \DTMdate{2021-5-14}. The gradient $m$ in the $0.5 \leqslant \lg{\rho} \leqslant 1.0$ range is calculated by the least-squares fit to $\lg{B}$ versus $\lg{\rho}$. 
In Fig.~\ref{fig:sbp_m}, the gradients of C/2019 L3 on each date are depicted, and the averaged value is indicated by a blue dotted line. 
Note that in Fig.~\ref{fig:sbp} the SBP is expressed as magnitude, and according to the relationship between magnitude and luminous intensity, the slope in such figure should be multiplied by \num{-0.4} to make the gradient $m$. The averaged $m$ is \num{-1.60}, and in most cases it is below \num{-1.5}, indicating that this comet's dust emission is in a nonsteady state. While for comet C/2020 P3, all images taken are so faint and it would bring about a great deal of uncertainty if we calculated its SBP. Moreover, the PSF artifacts resulting from imperfect PSF subtraction play a minor role on the SBP calculation. This is because the PSF stars are outside the SBP apertures, and the background is calculated in a more accurate manner. 

\begin{figure}
    \centering
    \includegraphics[width=\columnwidth]{sbp-210514-095.pdf}
    \caption{An example figure for R-band surface brightness of comet C/2019 L3 (ATLAS) on \DTMdate{2021-5-14}. The error plot is the SBP of comet, and the blue solid line is the SBP of a background star. The blue dotted line is a linear regression result of comet's SBP vs. $\lg{\rho}$ with $\lg{\rho}$ between 0.5 and 1.0, and the gradient $m$ related to the slope is marked on the graph with an arrow. }
    \label{fig:sbp}
\end{figure}

\begin{figure}
    \centering
    \includegraphics[width=\columnwidth]{sbp_m.pdf}
    \caption{The gradient $m$ of R-band surface brightness of comet C/2019 L3 (ATLAS) with date, blue dashed line is the averaged $m$, around which is a red shadow representing the error. }
    \label{fig:sbp_m}
\end{figure}


\subsection{$Af\rho$}

The $Af\rho$ value introduced by \citet{ahearn_comet_1984}, where $A$ represents the grain albedo, $f$ the filling factor, and $\rho$ the aperture, is commonly used to indicate the dust production activity of comets. Usually it is expressed in units of [\unit{\cm}] as equation (\ref{eq:afr}): 
\begin{equation}
    Af\rho = \frac{4 r^2 \Delta^2}{\rho} 10^{0.4(M_\odot - M_\mathrm{c})}, 
    \label{eq:afr}
\end{equation}
where $r$ is the heliocentric distance in units of [\unit{\astronomicalunit}], $\Delta$ the geocentric distance in units of [\unit{\km}], $\rho$ the aperture in units of [\unit{\km}], $M_\odot$ the absolute magnitude of the Sun (respectively \num{-26.13} and \num{-27.15} in B and R bands, see \citealt{willmer_absolute_2018}), and $M_\mathrm{c}$ the corresponding magnitude of comet under the aperture of $\rho$. 

Due to the phase darkening effect, it is necessary to adjust $Af\rho$ values at different phase angles to a specific angle. In this work, all observations were conducted at small phase angles, and we normalize the $Af\rho$ values to a phase angle of \ang{0} using equation (\ref{eq:a0frho}), as shown below:
\begin{equation}
    A(0)f\rho = \frac{A(\alpha)f\rho}{\phi(\alpha)}, \label{eq:a0frho}
\end{equation}
where $\alpha$ is the phase angle, and $\phi$ is the phase function. A composite phase function (see \citealt{schleicher_composition_2011, marcus_forward-scattering_2007}) suggested by D. Schleicher\footnote{\url{https://asteroid.lowell.edu/comet/dustphase.html}} is suitable for adjustion in this work. The related tabular data (\verb|dustphaseHM_table.txt|\footnote{\url{https://asteroid.lowell.edu/comet/dustphaseHM_table.html}}) provides the phase function with phase angle in the $\ang{0} \leqslant \alpha \leqslant \ang{180}$ range, and we adopt cubic spline interpolation method on it to obtain unlisted values. 

Fig.~\ref{fig:Afrho} shows some part of the R-band $A(0)f\rho$ profiles, and the $A(0)f\rho$ values calculated in the reference aperture with a radius of \qty{10000}{\km} are summarised in Table~\ref{tab:afrho}. When a comet possesses steady coma, its $Af\rho$ will be independent of aperture. In this paper, as is shown in Fig.~\ref{fig:Afrho}, that is not the case for comet C/2019 L3 or C/2020 P3, both of which reveal a steep increase in $Af\rho$ with the aperture $\rho$ near the comet center along with a smooth decrease with larger aperture. The increase results from usually the effect of seeing and observational circumstance, while the nonsteady dust emmission and possibly the fading or destruction of dust grain bring the decrease \citep{lara_behaviour_2003,tozzi_imaging_2003}.  

\begin{figure}
    \centering
    \subcaptionbox{R-band $A(0)f\rho$ of comet C/2019 L3 (ATLAS) on \DTMdate{2021-5-14}.}{
        \includegraphics[width=.98\columnwidth]{R-A0frho-C2019L3-210514-119.pdf}
    }

    \subcaptionbox{R-band $A(0)f\rho$ of comet C/2020 P3 (ATLAS) on \DTMdate{2021-4-2}.}{
        \includegraphics[width=.98\columnwidth]{R-A0frho-C2020P3-210402.pdf}
    }
    \caption{R-band $A(0)f\rho$ of comet C/2019 L3 (ATLAS) and C/2020 P3 (ATLAS). }
    \label{fig:Afrho}
\end{figure}

\begin{table}
    \centering
    \caption{$A(0)f\rho$ values for comets C/2019 L3 (ATLAS) and C/2020 P3 (ATLAS) within the aperture radius $\rho = \qty{10000}{\km}$. }\label{tab:afrho}
    \begin{threeparttable}
        % \resizebox{\linewidth}{!}{
        \begin{tabular}{cc}
            \toprule
            Observation Time & $A(0)f\rho$ [\unit{\cm}] ($\rho =$ \qty{10000}{\km}) \\
            \midrule
            \multicolumn{2}{l}{\textbf{C/2019 L3 (ATLAS)}} \\
            2021-03-28.729 & \num{13611 +- 1874} \\
            2021-04-02.739 & \num{9332 +- 539} \\
            2021-04-03.719 & \num{7959 +- 143} \\
            2021-04-04.691 & \num{6220 +- 69} \\
            2021-04-08.691 & \num{8985 +- 603} \\
            2021-04-12.724 & \num{7911 +- 102} \\
            2021-04-14.702 & \num{7320 +- 95} \\
            2021-04-15.714 & \num{5043 +- 244} \\
            2021-05-04.736 & \num{5943 +- 96} \\
            2021-05-10.735 & \num{7523 +- 134} \\
            2021-05-12.724 & \num{5701 +- 236} \\
            2021-05-14.721 & \num{6356 +- 152} \\
            \multicolumn{2}{l}{\textbf{C/2020 P3 (ATLAS)}} \\
            2021-04-02.775 & \num{869 +- 20} \\
            2021-05-11.744 & \num{708 +- 17} \\
            2021-05-12.749 & \num{606 +- 31} \\
            \bottomrule
        \end{tabular}
        % }
    \end{threeparttable}
\end{table}
\subsection{Coma Colors}

The photometric results in Table~\ref{tab:bvr} also summarises the color index of the two comets. 
Fig.~\ref{fig:color-r} is the color indices $\mathrm{B-V}$, $\mathrm{V-R}$, and $\mathrm{R-I}$ of C/2019 L3 as a function of heliocentric distance in units of [\unit{\astronomicalunit}], all exhibiting variability during the approach to perihelion. 
As C/2019 L3 approached perihelion, its $\mathrm{V-R}$ color index showed a tendency towards blue. 
Considering its position at a heliocentric distance of about {\qty{4}{\astronomicalunit}}, with a temperature of around {\qty{140}{\K}}, the gas-driven effects are significant. 
The colors for C/2019 L3 are as follows: 
$\langle \mathrm{B-V} \rangle = \num{0.75 +- 0.06}$, 
$\langle \mathrm{V-R} \rangle = \num{0.28 +- 0.05}$, and 
$\langle \mathrm{R-I} \rangle = \num{0.21 +- 0.05}$. 
Besides, the colors for C/2020 P3 are 
$\mathrm{B-V} = \num{1.04 +- 0.09}$, 
$\langle \mathrm{V-R} \rangle = \num{0.26 +- 0.05}$, and 
$\mathrm{R-I} = \num{0.76 +- 0.04}$.

\begin{figure}
    \centering
    \includegraphics[width=\linewidth]{color-r.pdf} 
    \caption{Color indices vs.\ heliocentric distance plot for comet C/2019 L3 (ATLAS).}\label{fig:color-r}
\end{figure}

Moreover, in order to indicate how the scattered color of the dusty coma varies with wavelengths, the reddening $\mathcal{R}$ (or normalized color) \citep{jewitt_cometary_1986, lara_behaviour_2003, mazzotta_epifani_dust_2011, shi_ccd_2015} is calculated in units of [\unit{\percent/\kilo\angstrom}], with the formula given as equation (\ref{eq:red}): 
\begin{equation}
\mathcal{R} = \frac{2}{Af\rho_1 + Af\rho_2} \frac{Af\rho_2 - Af\rho_1}{\lambda_2 - \lambda_1}, 
\label{eq:red}
\end{equation}
where $\lambda_1$ and $\lambda_2$ are the central wavelengths of the filters in units of [\unit{\nm}] (respectively the centers of B and R filters, {\qty{440}{\nm}} and {\qty{658}{\nm}}). With this parameter it is convenient to indicate the percentage of change in the strength of the continuum per {\qty{1000}{\angstrom}}. The results of dust reddening are summarised in Table~\ref{tab:reddening}. In order to avoid the possible effects from background residuals, it is calculated within the aperture radius of {\qty{20000}{\km}}. 

For comet C/2019 L3, the reddening undergoes significant variations over time, from positive value {\qty{13.75 +- 1.07}{\percent/\kilo\angstrom}} on \DTMdate{2021-3-28} to negative value {\qty{-15.69 +- 0.37}{\percent/\kilo\angstrom}} on \DTMdate{2021-4-12}. We plot the reddening as a function of date in Fig.~\ref{fig:reddening}, and the averaged value {\qty{0.94 +- 0.23}{\percent/\kilo\angstrom}} is marked as a horizontal blue dotted line. For comet C/2020 P3, only on \DTMdate{2021-5-12} is its observational data sufficient for the calculation of reddening with a result of {\qty{-6.65 +- 0.01}{\percent/\kilo\angstrom}}. 

The reddening of C/2019 L3 on each observation date is shown in Fig.~\ref{fig:reddening}, with the blue dotted line indicating the mean value. The plot reveals that over the observation course of nearly two months, C/2019 L3 exhibited multiple transitions from a reddish to a bluish color, possibly resulting from variations in the composition of the coma \citep{ivanova_colour_2017}. 

\begin{table}
    \centering
    \caption{Reddening of comets C/2019 L3 (ATLAS) and C/2020 P3 (ATLAS) within the aperture radius $\rho = \qty{20000}{\km}$. }\label{tab:reddening}
    \begin{threeparttable}
        \resizebox{\linewidth}{!}{
        \begin{tabular}{cccc}
            \toprule
            Observation Time & B-band $Af\rho$ & R-band $Af\rho$ & reddening [\unit{\percent/\kilo\angstrom}]\\
            \midrule
            \multicolumn{4}{l}{\textbf{C/2019 L3 (ATLAS)}} \\
            2021-03-28.729 & \num{9472 +- 753} & \num{12811 +- 1764} & \num{13.75 +- 1.07}\\
            2021-04-02.739 & \num{8763 +- 806} & \num{10719 +- 617} & \num{9.21 +- 0.50}\\
            2021-04-03.719 & \num{9886 +- 274} & \num{9538 +- 156} & \num{-1.65 +- 0.03}\\
            2021-04-04.691 & \num{9501 +- 343} & \num{8734 +- 93} & \num{-3.86 +- 0.07}\\
            2021-04-08.691 & \num{9288 +- 402} & \num{10797 +- 725} & \num{6.89 +- 0.27}\\
            2021-04-12.724 & \num{12683 +- 602} & \num{8979 +- 113} & \num{-15.69 +- 0.37}\\
            2021-04-14.702 & \num{8809 +- 382} & \num{8700 +- 110} & \num{-0.57 +- 0.01}\\
            2021-04-15.714 & \num{6077 +- 283} & \num{7173 +- 271} & \num{7.59 +- 0.23}\\
            2021-05-04.736 & \num{7724 +- 264} & \num{8065 +- 130} & \num{1.98 +- 0.04}\\
            2021-05-10.735 & \num{9030 +- 285} & \num{8891 +- 154} & \num{-0.71 +- 0.01}\\
            2021-05-12.724 & \num{9044 +- 336} & \num{7958 +- 326} & \num{-5.86 +- 0.16}\\
            2021-05-14.721 & \num{8428 +- 258} & \num{8475 +- 199} & \num{0.25 +- 0.01}\\
            \multicolumn{4}{l}{\textbf{C/2020 P3 (ATLAS)}} \\
            2021-05-12.749 & \num{934 +- 74} & \num{799 +- 37} & \num{-6.65 +- 0.01}\\
            \bottomrule
        \end{tabular}
        }
    \end{threeparttable}
\end{table}

\begin{figure}
    \centering
    \includegraphics[width=\columnwidth]{reddening.pdf}
    \caption{The reddening of comet C/2019 L3 (ATLAS) with date, blue dashed line is the averaged reddening, and the surrounding red shadow shows the error. }\label{fig:reddening}
\end{figure}
