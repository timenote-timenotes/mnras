\section{Observations and data reduction} \label{sec:obs_data}

\subsection{Observations}

The observations on two comets metioned above were conducted using three different telescopes belonging to the International Scientific Optical Network (ISON) between 2021 March and 2021 May, namely the {\qty{1.0}{\m}} Zeiss-1000\footnote{\href{https://link.springer.com/content/pdf/10.1134/S1990341320040112.pdf}{https://link.springer.com/content/pdf/10.1134/S1990341320040112.pdf}} 
telescope at Simeiz Observatory (Simeiz), the {\qty{2.6}{\m}} Shajn Telescope (ZTSh\footnote{\href{https://crao.ru/index.php/en/telescopes-en/ztsh-en}{https://crao.ru/index.php/en/telescopes-en/ztsh-en}}) 
at Crimean Astrophysical Observatory (CrAO), and the {\qty{0.7}{\m}} Maksutov meniscus telescope at Abastumani Astrophysical Observatory (AbAO\footnote{\href{https://www.oato.inaf.it/blazars/webt/abastumani-astrophysical-observatory-georgia-fsu/}{https://www.oato.inaf.it/blazars/webt/abastumani-astrophysical-observatory-georgia-fsu/}}). 
The technical parameters of these telescopes are listed in Table~\ref{tab:telescope}. Four kinds of broadband observational data were obtained with broadband B, V, R and I filters in the Johnson-Cousins system. The log of all observations is presented in Table~\ref{tab:c2019andc2020}. Most of the data were obtained from telescope Maksutov. For comet C/2019 L3, the observations were carried out before the time of perihelion, while for comet C/2020 P3, it was near its perihelion during the observation period. Besides, some of the data shown in Table~\ref{tab:c2019andc2020} is not used in subsequent analysis due to its faintness. 

% 使用的望远镜信息
\begin{table}
    \centering
    \caption{Information of instrunments used. }\label{tab:telescope}
    \begin{threeparttable}
        \resizebox{\linewidth}{!}{
        \begin{tabular}{ccccccc}
            \toprule
            Telescope & CCD Camera & Focal length [\unit{\mm}] & Frame size & Pixel size & Site & MPC Code\tnote{1} \\
            \midrule
            ZEISS-1000 & FLI & \num{13000} & \makecell[c]{ \qtyproduct{2048 x 2048}{px} \\ $ \ang{;7.3;} \times \ang{;7.3;} $ } & \makecell[c]{\qty{13.5}{\um} \\ \qty{0.216}{^{\prime \prime}/px}}  & Crimea-Simeïs & 094 \\
            ZTSh & FLI PL-4240 & \num{10000} & \makecell[c]{ \qtyproduct{2048 x 2048}{px} \\ $ \ang{;14.2;} \times \ang{;14.2;} $ } & \makecell[c]{ \qty{13.5}{\um} \\ \qty{0.4}{^{\prime \prime}/px}} & Crimea-Nauchnij & 095 \\
            Maksutov & CCD FLI PL4240 & \num{2141} & \qtyproduct{2048 x 2048}{px} & \qty{13.5}{\um} & Abastuman       & 119 \\
            \bottomrule
        \end{tabular}
        }
        \begin{tablenotes}
            \item[1] see Minor Planet Center Observatory Code at \\
            \href{www.minorplanetcenter.net/iau/lists/ObsCodesF.html}{www.minorplanetcenter.net/iau/lists/ObsCodesF.html}
        \end{tablenotes}
    \end{threeparttable}
\end{table}

% 两颗彗星观测数据的总览
\begin{table}
    \centering
    \caption{Log of observations on C/2019 L3 and C/2020 P3}\label{tab:c2019andc2020}
    \begin{threeparttable}
        \resizebox{\linewidth}{!}{
        \begin{tabular}{cccccccc}
            \toprule
            Observation Date & r\tnote{1}~[\si{\astronomicalunit}] & $\Delta$\tnote{2}~[\si{\astronomicalunit}] & Ph.A\tnote{3} & Filters\tnote{4} & Size\tnote{5}~[\si{px}] & Scale\tnote{6}~[\si{\km/px}] & Telescope \\
            \midrule
            \multicolumn{8}{l}{\textbf{C/2019 L3}} \\
            2021-03-28 & 4.385 & 4.916 & 10.4 & B$\times$10, V$\times$10, R$\times$10 & 1018 $\times$ 1018 & 2075 & ZEISS-1000 \\
            2021-04-02 & 4.360 & 4.933 & 10.1 & B$\times$4, V$\times$4, R$\times$5 & 2048 $\times$ 2048 & 4654 & Maksutov \\
            2021-04-03 & 4.355 & 4.936 & 10.1 & B$\times$5, V$\times$5, R$\times$6 & 2048 $\times$ 2048 & 4658 & Maksutov \\
            2021-04-04 & 4.350 & 4.939 & 10.0 & B$\times$10, V$\times$10, R$\times$10 & 2048 $\times$ 2048 & 4660 & Maksutov \\
            2021-04-08 & 4.331 & 4.950 & 9.7 & B$\times$6, V$\times$5, R$\times$5 & 2048 $\times$ 2048 & 4670 & Maksutov \\
            2021-04-12 & 4.311 & 4.960 & 9.5 & B$\times$4, V$\times$5, R$\times$3 & 2048 $\times$ 2048 & 4684 & Maksutov \\
            2021-04-14 & 4.301 & 4.965 & 9.3 & B$\times$4, V$\times$5, R$\times$5 & 2048 $\times$ 2048 & 4686 & Maksutov \\
            2021-04-15 & 4.296 & 4.967 & 9.3 & B$\times$4, V$\times$4, R$\times$4 & 2048 $\times$ 2048 & 4687 & Maksutov \\
            2021-05-04 & 4.205 & 4.987 & 8.0 & B$\times$10, V$\times$10, R$\times$10, I$\times$10 & 1365 $\times$ 1365 & 1587 & ZEISS-1000 \\
            2021-05-10 & 4.177 & 4.985 & 7.6 & B$\times$3, V$\times$3, R$\times$3, I$\times$3 & 2048 $\times$ 2048 & 4709 & Maksutov \\
            2021-05-12 & 4.168 & 4.984 & 7.5 & B$\times$5, V$\times$5, R$\times$5, I$\times$5 & 2048 $\times$ 2048 & 4706 & Maksutov \\
            2021-05-14 & 4.159 & 4.982 & 7.4 & \makecell[c]{B$\times$11, V$\times$11, R$\times$12, I$\times$11 \\ B$\times$7, V$\times$7, R$\times$7, I$\times$7} & \makecell[c]{1024 $\times$ 1024 \\ 2048 $\times$ 2048}  & \makecell[c]{2015 \\ 4706} & \makecell[c]{ZTSh \\ Maksutov} \\

            \multicolumn{8}{l}{\textbf{C/2020 P3}} \\
            2021-03-28 & 6.956 & 6.814 & 8.2 & B$\times$7, V$\times$7, R$\times$7 & 1018 $\times$ 1018 & 2937 & ZEISS-1000 \\
            2021-04-02 & 6.990 & 6.813 & 8.2 & V$\times$11, R$\times$14 & 2048 $\times$ 2048 & 6592 & Maksutov \\
            2021-05-11 & 7.216 & 6.814 & 7.6 & C$\times$4, V$\times$5, R$\times$5 & 2048 $\times$ 2048 & 6807 & Maksutov \\
            2021-05-12 & 7.221 & 6.814 & 7.6 & \makecell[c]{B$\times$3, V$\times$3, R$\times$3, I$\times$3 \\ B$\times$6, V$\times$5, R$\times$5, I$\times$5} & \makecell[c]{1024 $\times$ 1024 \\ 2048 $\times$ 2048}  & \makecell[c]{2914 \\ 6807} & \makecell[c]{ZTSh \\ Maksutov} \\
            \bottomrule
        \end{tabular}
        }
        \begin{tablenotes}
            \item[1] heliocentric distance
            \item[2] geocentric distance
            \item[3] phase angle
            \item[4] numbers after `$\times$' indicate amounts of images under corresponding filters, \\
            `C' means `clear glass'
            \item[5] size of original image
            \item[6] the calculated image scale in \si{\km} per pixel
        \end{tablenotes}
    \end{threeparttable}
\end{table}

\subsection{Data reduction}

All of the images had been corrected with common methods for dark subtraction, bias subtraction and flat-field normalization when ISON provided the observational data for this research. In order to raise the signal-to-noise ratio, images on different observational dates were aligned and sum combined for different filters according to the photocenter of comet. The same thing was also done on the basis of photocenter of background star as the motion of comet in each frame is obvious. Several tasks in IRAF\footnote{\href{https://github.com/iraf-community/iraf}{https://github.com/iraf-community/iraf}}, i.e., the \textbf{I}mage \textbf{R}eduction and \textbf{A}nalysis \textbf{F}acility, were applied for this procedure, including \texttt{center}, \verb|imshift| and \verb|combine|. Since the comet and reference stars for photometry are in the same frame, the airmass correction is not necessary. Given that some images are in the case where comet gets too close to the background stars around, psf subtraction was conducted to cover up these stars with tasks such as \verb|substar|. 

With several photometric tasks in IRAF, the instrunment magnitudes of two comets were calculated, and so were the comparison stars in each frame. Then the apparent magnitudes were obtained by differential photometry. The photometric calibration of these two comets' data was performed with the UCAC4 \citep{zacharias_fourth_2013} and UCAC5 \citep{zacharias_ucac5_2017} catalogues in 
Aladin\footnote{\href{http://aladin.u-strasbg.fr}{http://aladin.u-strasbg.fr}} 
where positions, B, V, and R magnitudes for comparison stars can be queryed easily. Additionally, the GCVS 5.1 \citep{samus_general_2017} catalogue was applied to inspect the variability of all the comparison stars. 

In order to measure the  instrunment magnitudes of comets and comparison stars, circular aperture photometry was applied. The photometric aperture of comparison stars was determined by nearly \si{\num{2}} times the full-wide at half-maximum (FWHM). Since three different telescopes were involved during the observation, the photometry of comets was performed on different apertures centered on the comet photocenter, with all of them ensured to be up to \si{\num{1.5}} times the full-wide at half-maximum and close to each other in arcsecond as much as possible. 
The ranges of seeings for the three telescopes are as follows: \ang{;;1.2} to \ang{;;7.5} for ZEISS-1000, \ang{;;3.9} to \ang{;;11.7} for Maksutov, and \ang{;;2.4} to \ang{;;3.9} for ZTSh. 
The photometric results are listed as Table~\ref{tab:bvr}, and the error of photometry $e_{\mathrm{phot}}$ is derived from equation (\ref{eq:err}) as follows: 
\begin{equation}
    e_{\mathrm{phot}} = \sqrt{e_{\mathrm{c}}^{2} + \sigma_\mathrm{s}^2}, 
    \label{eq:err}
\end{equation}
where $e_\mathrm{c}$ is the magnitude error of comet from IRAF photometric file and $\sigma_\mathrm{s}$ is the standard deviation of the calibration values in differential photometry. 






