% mnras_template.tex 
%
% LaTeX template for creating an MNRAS paper
%
% v3.0 released 14 May 2015
% (version numbers match those of mnras.cls)
%
% Copyright (C) Royal Astronomical Society 2015
% Authors:
% Keith T. Smith (Royal Astronomical Society)

% Change log
%
% v3.0 May 2015
%    Renamed to match the new package name
%    Version number matches mnras.cls
%    A few minor tweaks to wording
% v1.0 September 2013
%    Beta testing only - never publicly released
%    First version: a simple (ish) template for creating an MNRAS paper

%%%%%%%%%%%%%%%%%%%%%%%%%%%%%%%%%%%%%%%%%%%%%%%%%%
% Basic setup. Most papers should leave these options alone.
\documentclass[fleqn,usenatbib]{mnras}

% MNRAS is set in Times font. If you don't have this installed (most LaTeX
% installations will be fine) or prefer the old Computer Modern fonts, comment
% out the following line
\usepackage{newtxtext,newtxmath}
% Depending on your LaTeX fonts installation, you might get better results with one of these:
% \usepackage{mathptmx}
% \usepackage{txfonts}

% Use vector fonts, so it zooms properly in on-screen viewing software
% Don't change these lines unless you know what you are doing
\usepackage[T1]{fontenc}

% Allow "Thomas van Noord" and "Simon de Laguarde" and alike to be sorted by "N" and "L" etc. in the bibliography.
% Write the name in the bibliography as "\VAN{Noord}{Van}{van} Noord, Thomas"
\DeclareRobustCommand{\VAN}[3]{#2}
\let\VANthebibliography\thebibliography
\def\thebibliography{\DeclareRobustCommand{\VAN}[3]{##3}\VANthebibliography}


%%%%% AUTHORS - PLACE YOUR OWN PACKAGES HERE %%%%%

% Only include extra packages if you really need them. Common packages are:
\usepackage{graphicx}	% Including figure files
\usepackage{amsmath}	% Advanced maths commands
% \usepackage{amssymb}	% Extra maths symbols

\usepackage{caption}
\usepackage{subcaption}
\graphicspath{{figures/}}
\usepackage{verbatim}
\usepackage{booktabs}
\usepackage{threeparttable}
\usepackage{makecell}

\usepackage{siunitx}
\sisetup{
	separate-uncertainty = true,
	% inter-unit-product = \ensuremath{{}\cdot{}}
}
\DeclareSIUnit\angstrom{\text{\AA}}

\usepackage[version=4]{mhchem}
\usepackage[calc]{datetime2}
% \usepackage{datetime2-calc}
\DTMnewdatestyle{mnras}{  % set the date format for mnras
    \renewcommand{\DTMdisplaydate}[4]{##1 \pgfcalendarmonthname{##2} \number##3 }
    \renewcommand{\DTMDisplaydate}{\DTMdisplaydate}
}
\DTMsetdatestyle{mnras}
% \DTMtryregional{en}{GB}
% \DTMsetdatestyle{default} % default iso yyyymd ddmmyyyy
% \DTMsetup{datesep={/}}
\usepackage{soul}
\setulcolor{blue}
\setstcolor{red}
\sethlcolor{yellow}
\soulregister\cite7
\soulregister\ref7
\soulregister\autoref7
\soulregister\citep7
\soulregister\citet7
\soulregister\citealp7
\soulregister\citealt7
\soulregister\pageref7
\soulregister{\textbf}{7}
\soulregister{\href}{1}
\soulregister{\si}{7}
\soulregister{\qty}{1}
\soulregister{\SI}{1}
\soulregister{\ang}{7}
\soulregister{\num}{7}
\soulregister{\ce}{7}
\soulregister\DTMdate7
\soulregister\texttt7

\newcommand*{\dif}{\mathop{}\!\mathrm{d}}

% \usepackage{showframe} 	% some manual settings

%%%%%%%%%%%%%%%%%%%%%%%%%%%%%%%%%%%%%%%%%%%%%%%%%%

%%%%% AUTHORS - PLACE YOUR OWN COMMANDS HERE %%%%%

% Please keep new commands to a minimum, and use \newcommand not \def to avoid
% overwriting existing commands. Example:
%\newcommand{\pcm}{\,cm$^{-2}$}	% per cm-squared

%%%%%%%%%%%%%%%%%%%%%%%%%%%%%%%%%%%%%%%%%%%%%%%%%%

%%%%%%%%%%%%%%%%%%% TITLE PAGE %%%%%%%%%%%%%%%%%%%

% Title of the paper, and the short title which is used in the headers.
% Keep the title short and informative.
\title[Short title, max. 45 characters]{Broadband CCD photometry of Long-period comets C/2019 L3 (ATLAS) and C/2020 P3 (ATLAS)}

% The list of authors, and the short list which is used in the headers.
% If you need two or more lines of authors, add an extra line using \newauthor
\author[Shaofeng Sun et al.]{
Shaofeng Sun,$^{1, 2, 3}$
Jianchun Shi,$^{1, 2, 3, 4}$\thanks{E-mail: jcshi@pmo.ac.cn (JS)}
and Yuehua Ma$^{1, 2, 3}$\thanks{E-mail: yhma@pmo.ac.cn (YM)}
\\
% List of institutions
$^{1}$Purple Mountain Observatory, Chinese Academy of Sciences, Nanjing 210023, China\\
$^{2}$Deep Space Exploration Laboratory / School of Astronomy and Space Science, University of Science and Technology of China, Hefei 230026, China\\
$^{3}$Key Laboratory of Planetary Sciences, Chinese Academy of Sciences, Nanjing 210023, China\\
$^{4}$CAS Center for Excellence in Comparative Planetology, Hefei 230026, China
}

% These dates will be filled out by the publisher
\date{Accepted XXX. Received YYY; in original form ZZZ}

% Enter the current year, for the copyright statements etc.
\pubyear{\DTMfetchyear{now}}

% Don't change these lines
\begin{document}
\label{firstpage}
\pagerange{\pageref{firstpage}--\pageref{lastpage}}
\maketitle

\DTMsavenow{now}

% Abstract of the paper
\begin{abstract}
	The observational results for two \ul{Long-period} comets, namely C/2019 L3 and C/2020 P3, are analyzed in this paper. 
	Broadband B-, V-, R- and I-images were obtained with three telescopes: ZEISS-1000 at Simeiz Observatory, Maksutov at Abastumani Astrophysical Observatory and ZTSh at Crimean Astrophysical Observatory. 
	The two objects were observed between \DTMdate{2021-3-28} and \DTMdate{2021-5-14} for a total of 13 nights. 
	We study the morphology, photometry, surface brightness, $Af\rho$ values as well as coma colors of the two comets based on these data. 
\end{abstract}

% Select between one and six entries from the list of approved keywords.
% Don't make up new ones.
\begin{keywords}
Long-period comets -- photometric -- data analysis
\end{keywords}

%%%%%%%%%%%%%%%%%%%%%%%%%%%%%%%%%%%%%%%%%%%%%%%%%%

%%%%%%%%%%%%%%%%% BODY OF PAPER %%%%%%%%%%%%%%%%%%

\section{Introduction}

This is a simple template for authors to write new MNRAS papers.
See \texttt{mnras\_sample.tex} for a more complex example, and \texttt{mnras\_guide.tex}
for a full user guide.

All papers should start with an Introduction section, which sets the work
in context, cites relevant earlier studies in the field by \citet{Fournier1901},
and describes the problem the authors aim to solve \citep[e.g.][]{vanDijk1902}.
Multiple citations can be joined in a simple way like \citet{deLaguarde1903, delaGuarde1904}.



\section{Observations and data reduction} \label{sec:obs_data}

\subsection{Observations}

The observations on two comets metioned above were conducted using three different telescopes belonging to the International Scientific Optical Network (ISON) between 2021 March and 2021 May, namely the {\SI{1.0}{\m}} Zeiss-1000\footnote{\href{https://link.springer.com/content/pdf/10.1134/S1990341320040112.pdf}{https://link.springer.com/content/pdf/10.1134/S1990341320040112.pdf}} 
telescope at Simeiz Observatory (Simeiz), the {\SI{2.6}{\m}} Shajn Telescope (ZTSh\footnote{\href{https://crao.ru/index.php/en/telescopes-en/ztsh-en}{https://crao.ru/index.php/en/telescopes-en/ztsh-en}}) 
at Crimean Astrophysical Observatory (CrAO), and the {\SI{0.7}{\m}} Maksutov meniscus telescope at Abastumani Astrophysical Observatory (AbAO\footnote{\href{https://www.oato.inaf.it/blazars/webt/abastumani-astrophysical-observatory-georgia-fsu/}{https://www.oato.inaf.it/blazars/webt/abastumani-astrophysical-observatory-georgia-fsu/}}). 
The technical parameters of these telescopes are listed in Table~\ref{tab:telescope}. Four kinds of broadband observational data were obtained with broadband B, V, R and I filters in the Johnson-Cousins system. The log of all observations is presented in Table~\ref{tab:c2019andc2020}. Most of the data were obtained from telescope Maksutov. For comet C/2019 L3, the observations were \ul{carried out} before the time of perihelion, while for comet C/2020 P3, it \st{reached} \ul{was near} its perihelion during the observation period. Besides, some of the data shown in Table~\ref{tab:c2019andc2020} is not used in subsequent analysis due to its faintness. 

% 使用的望远镜信息
\begin{table}
    \centering
    \caption{Information of instrunments used. }\label{tab:telescope}
    \begin{threeparttable}
        \resizebox{\linewidth}{!}{
        \begin{tabular}{ccccccc}
            \toprule
            Telescope & CCD Camera & Focal length [\si{\mm}] & Frame size & Pixel size & Site & MPC Code\tnote{1} \\
            \midrule
            ZEISS-1000 & FLI & \num{13000} & \makecell[c]{ \SI{2048 x 2048}{px} \\ $ \ang{;7.3;} \times \ang{;7.3;} $ } & \makecell[c]{\SI{13.5}{\um} \\ \SI{0.216}{^{\prime \prime}/px}}  & Crimea-Simeïs & 094 \\
            ZTSh & FLI PL-4240 & \num{10000} & \makecell[c]{ \SI{2048 x 2048}{px} \\ $ \ang{;14.2;} \times \ang{;14.2;} $ } & \makecell[c]{ \SI{13.5}{\um} \\ \SI{0.4}{^{\prime \prime}/px}} & Crimea-Nauchnij & 095 \\
            Maksutov & CCD FLI PL4240 & \num{2141} & \SI{2048 x 2048}{px} & \SI{13.5}{\um} & Abastuman       & 119 \\
            \bottomrule
        \end{tabular}
        }
        \begin{tablenotes}
            \item[1] see Minor Planet Center Observatory Code at \\
            \href{www.minorplanetcenter.net/iau/lists/ObsCodesF.html}{www.minorplanetcenter.net/iau/lists/ObsCodesF.html}
        \end{tablenotes}
    \end{threeparttable}
\end{table}

% 两颗彗星观测数据的总览
\begin{table}
    \centering
    \caption{Log of observations on C/2019 L3 and C/2020 P3}\label{tab:c2019andc2020}
    \begin{threeparttable}
        \resizebox{\linewidth}{!}{
        \begin{tabular}{cccccccc}
            \toprule
            Observation Date & r\tnote{1}~[\si{\astronomicalunit}] & $\Delta$\tnote{2}~[\si{\astronomicalunit}] & Ph.A\tnote{3} & Filters\tnote{4} & Size\tnote{5}~[\si{px}] & Scale\tnote{6}~[\si{\km/px}] & Telescope \\
            \midrule
            \multicolumn{8}{l}{\textbf{C/2019 L3}} \\
            2021-03-28 & 4.385 & 4.916 & 10.4 & B$\times$10, V$\times$10, R$\times$10 & 1018 $\times$ 1018 & 2075 & ZEISS-1000 \\
            2021-04-02 & 4.360 & 4.933 & 10.1 & B$\times$4, V$\times$4, R$\times$5 & 2048 $\times$ 2048 & 4654 & Maksutov \\
            2021-04-03 & 4.355 & 4.936 & 10.1 & B$\times$5, V$\times$5, R$\times$6 & 2048 $\times$ 2048 & 4658 & Maksutov \\
            2021-04-04 & 4.350 & 4.939 & 10.0 & B$\times$10, V$\times$10, R$\times$10 & 2048 $\times$ 2048 & 4660 & Maksutov \\
            2021-04-08 & 4.331 & 4.950 & 9.7 & B$\times$6, V$\times$5, R$\times$5 & 2048 $\times$ 2048 & 4670 & Maksutov \\
            2021-04-12 & 4.311 & 4.960 & 9.5 & B$\times$4, V$\times$5, R$\times$3 & 2048 $\times$ 2048 & 4684 & Maksutov \\
            2021-04-14 & 4.301 & 4.965 & 9.3 & B$\times$4, V$\times$5, R$\times$5 & 2048 $\times$ 2048 & 4686 & Maksutov \\
            2021-04-15 & 4.296 & 4.967 & 9.3 & B$\times$4, V$\times$4, R$\times$4 & 2048 $\times$ 2048 & 4687 & Maksutov \\
            2021-05-04 & 4.205 & 4.987 & 8.0 & B$\times$10, V$\times$10, R$\times$10, I$\times$10 & 1365 $\times$ 1365 & 1587 & ZEISS-1000 \\
            2021-05-10 & 4.177 & 4.985 & 7.6 & B$\times$3, V$\times$3, R$\times$3, I$\times$3 & 2048 $\times$ 2048 & 4709 & Maksutov \\
            2021-05-12 & 4.168 & 4.984 & 7.5 & B$\times$5, V$\times$5, R$\times$5, I$\times$5 & 2048 $\times$ 2048 & 4706 & Maksutov \\
            2021-05-14 & 4.159 & 4.982 & 7.4 & \makecell[c]{B$\times$11, V$\times$11, R$\times$12, I$\times$11 \\ B$\times$7, V$\times$7, R$\times$7, I$\times$7} & \makecell[c]{1024 $\times$ 1024 \\ 2048 $\times$ 2048}  & \makecell[c]{2015 \\ 4706} & \makecell[c]{ZTSh \\ Maksutov} \\

            \multicolumn{8}{l}{\textbf{C/2020 P3}} \\
            2021-03-28 & 6.956 & 6.814 & 8.2 & B$\times$7, V$\times$7, R$\times$7 & 1018 $\times$ 1018 & 2937 & ZEISS-1000 \\
            2021-04-02 & 6.990 & 6.813 & 8.2 & V$\times$11, R$\times$14 & 2048 $\times$ 2048 & 6592 & Maksutov \\
            2021-05-11 & 7.216 & 6.814 & 7.6 & C$\times$4, V$\times$5, R$\times$5 & 2048 $\times$ 2048 & 6807 & Maksutov \\
            2021-05-12 & 7.221 & 6.814 & 7.6 & \makecell[c]{B$\times$3, V$\times$3, R$\times$3, I$\times$3 \\ B$\times$6, V$\times$5, R$\times$5, I$\times$5} & \makecell[c]{1024 $\times$ 1024 \\ 2048 $\times$ 2048}  & \makecell[c]{2914 \\ 6807} & \makecell[c]{ZTSh \\ Maksutov} \\
            \bottomrule
        \end{tabular}
        }
        \begin{tablenotes}
            \item[1] heliocentric distance
            \item[2] geocentric distance
            \item[3] phase angle
            \item[4] numbers after `$\times$' indicate amounts of images under corresponding filters, \\
            `C' means `clear glass'
            \item[5] size of original image
            \item[6] the calculated image scale in \si{\km} per pixel
        \end{tablenotes}
    \end{threeparttable}
\end{table}

\subsection{Data reduction}

All of the images had been corrected with common methods for dark subtraction, bias subtraction and flat-field normalization when ISON provided the observational data for this research. In order to raise the signal-to-noise ratio, images on different observational dates were aligned and sum combined for different filters according to the photocenter of comet. The same thing was also done on the basis of photocenter of background star as the motion of comet in each frame is obvious. Several tasks in IRAF\footnote{\href{https://github.com/iraf-community/iraf}{https://github.com/iraf-community/iraf}}, i.e., the \textbf{I}mage \textbf{R}eduction and \textbf{A}nalysis \textbf{F}acility, were applied for this procedure, including \texttt{center}, \verb|imshift| and \verb|combine|. Since the comet and reference stars for photometry are in the same frame, the airmass correction is not necessary. Given that some images are in the case where comet gets too close to the background stars around, psf subtraction was conducted to cover up these stars with tasks such as \verb|substar|. 

With several photometric tasks in IRAF, the instrunment magnitudes of two comets were calculated, and so were the comparison stars in each frame. Then the apparent magnitudes were obtained by differential photometry. The photometric calibration of these two comets' data was performed with the UCAC4 \citep{zacharias_fourth_2013} and UCAC5 \citep{zacharias_ucac5_2017} catalogues in 
Aladin\footnote{\href{http://aladin.u-strasbg.fr}{http://aladin.u-strasbg.fr}} 
where positions, B, V, and R magnitudes for comparison stars can be queryed easily. Additionally, the GCVS 5.1 \citep{samus_general_2017} catalogue was applied to inspect the variability of all the comparison stars. 

In order to measure the  instrunment magnitudes of comets and comparison stars, circular aperture photometry was applied. The photometric aperture of comparison stars was determined by nearly \si{\num{2}} times the full-wide at half-maximum (FWHM). Since three different telescopes were involved during the observation, the photometry of comets was performed on different apertures centered on the comet photocenter, with all of them ensured to be up to \si{\num{1.5}} times the full-wide at half-maximum and close to each other in arcsecond as much as possible. 
\ul{ % 描述3台望远镜下恒星seeing的范围
    The ranges of seeings for the three telescopes are as follows: \ang{;;1.2} to \ang{;;7.5} for ZEISS-1000, \ang{;;3.9} to \ang{;;11.7} for Maksutov, and \ang{;;2.4} to \ang{;;3.9} for ZTSh. 
} 
The photometric results are listed as Table~\ref{tab:bvr}, and the error of photometry $e_{\mathrm{phot}}$ is derived from equation (\ref{eq:err}) as follows: 
\begin{equation}
    e_{\mathrm{phot}} = \sqrt{e_{c}^{2} + \sigma_s^2}, 
    \label{eq:err}
\end{equation}
where $e_c$ is the magnitude error of comet from IRAF photometric file and $\sigma_s$ is the standard deviation of the calibration values in differential photometry. 








\section{Results} \label{sec:res}

\subsection{Morphology}

The original images of C/2019 L3 in the R band exhibit greater clarity compared to images in other bands, and they all resemble a star-like appearance. On the other hand, the original images of C/2020 P3 appear very faint, and the ones taken in the R band are also relatively clearer than those in other bands. Unlike C/2019 L3, some images of C/2020 P3 reveal the presence of an extended tail. After applying the data reduction process mentioned before, we get one image for each filter on each observational date. Fig.~\ref{fig:combinedimg} is a thumbnail view of the reduction result with comet located in center of each small parts whose field of view are all $\ang{;2;} \times \ang{;2;}$. Two different telescopes were involved in observing comet C/2019 L3 on \DTMdate{2021-5-14}, thus in Fig.~\ref{fig:combinedimg} there are two groups of thunbnails on this date with the upper one by telescope ZTSh and the lower one by telescope Maksutov, and the same applies to comet C/2020 P3 on \DTMdate{2021-5-12}. The black spots in several part of this view result from the psf subtraction process, as it is not always perfect for star subtraction. For comet C/2019 L3, the observational record is abundant. The I band filter was added in the observation from 2021 May, providing more data for this study. From the combined images of C/2019 L3, it is easily to notice that this comet is very round and seemingly isotropic. For comet C/2020 P3, the image resolution is lower and it is not easily discernible. Nevertheless, its extended coma looking like a tail is obvious, especially for the images taken on \DTMdate{2021-5-12}, which roughly measured \ang{;;20} from comet center in R-band image. 

For morphology analysis, it is necessasy to apply some image enhancement techniques with which we can recognize some features and structures hidden in coma \citep{samarasinha_image_2014}. Many pieces of software have been developed for comet image enhancement, such as Astroart 8\footnote{\url{https://www.msb-astroart.com/down_en.htm}} and online tool Cometary Coma Image Enhancement Facility\footnote{\url{https://www.psi.edu/research/cometimen}}. In this work, 
we applied azimuthal renormalization method on images of C/2019 L3 on \DTMdate{2021-5-14} by telescope Maksutov since other images are too faint and show no evident features after enhanced. The result is shown in Fig.~\ref{fig:aziren}, where the enhanced images under V and R filters show a small northeastward fan-shape structure, while the enhanced images under B and I filters give no obvious feature.  

\begin{figure}
    \centering
    \subcaptionbox{C/2019 L3 \added{(ATLAS)}}[\linewidth]{
        \includegraphics[width=.9\linewidth]{combine1.pdf}
        \includegraphics[width=.9\linewidth]{combine2.pdf} 
        \includegraphics[width=.9\linewidth]{combine3.pdf}
        \includegraphics[width=.9\linewidth]{combine4.pdf}
    }
    \caption{The processed images of comets C/2019 L3 \added{(ATLAS)} and C/2020 P3 \added{(ATLAS)} in different filters, with north at the top and east to the left. The field of view is $ 2^{\prime} \times 2^{\prime} $ for each thumbnail. Blue arrow shows the direction of the comets' velocity, and the orange arrow shows the direction to the Sun. }
    \label{fig:combinedimg}
\end{figure}

\begin{figure}
    \centering
    \ContinuedFloat
    \subcaptionbox{C/2019 L3 \added{(ATLAS)}}[\linewidth]{
        \includegraphics[width=\linewidth]{combine5.pdf}
        \includegraphics[width=\linewidth]{combine6.pdf} 
        \includegraphics[width=\linewidth]{combine7.pdf} 
    }
    \caption{(Continued)}
\end{figure}

\begin{figure}
    \centering
    \ContinuedFloat
    \subcaptionbox{C/2020 P3 \added{(ATLAS)}}[\linewidth]{
        \includegraphics[width=\linewidth]{combine8.pdf}
        \includegraphics[width=\linewidth]{combine9.pdf} 
    }
    \caption{(Continued)}
\end{figure}

\begin{figure}
    \centering
    \includegraphics[width=\linewidth]{azi_ren.pdf}
    \caption{Azimuthal-renormalized image of C/2019 L3 \added{(ATLAS)} on \DTMdate{2021-5-14}, the field of view for each thunbnail is also $\ang{;2;}\times\ang{;2;}$. \label{fig:aziren}}
\end{figure}
    
% BVR
\begin{table}
    \centering
    \caption{Photometric results of comets C/2019 L3 \added{(ATLAS)} and C/2020 P3 \added{(ATLAS)}. }\label{tab:bvr}
    \begin{threeparttable}
        \resizebox{\linewidth}{!}{
        \begin{tabular}{ccccccccc}
            \toprule
            Observation Time\tnote{1} & $\rho\tnote{2}~[^{\prime \prime}]$ & B & V & R & I & $\mathrm{B}-\mathrm{V}$ & $\mathrm{V}-\mathrm{R}$ & $\mathrm{R}-\mathrm{I}$\\
            \midrule
            \multicolumn{9}{l}{\textbf{C/2019 L3}} \\
            2021-03-28.729 & \num{14.5} & \num{14.32 +- 0.09} & \num{13.72 +- 0.07} & \num{13.17 +- 0.15} & - & \num{0.60 +- 0.11} & \num{0.55 +- 0.16} & - \\
            2021-04-02.739 & \num{15.6} & \num{14.37 +- 0.10} & \num{13.62 +- 0.04} & \num{13.14 +- 0.06} & - & \num{0.75 +- 0.11} & \num{0.48 +- 0.07} & - \\
            2021-04-03.719 & \num{15.6} & \num{14.25 +- 0.05} & \num{13.55 +- 0.03} & \num{13.08 +- 0.07} & - & \num{0.70 +- 0.05} & \num{0.47 +- 0.08} & - \\
            2021-04-04.691 & \num{15.6} & \num{14.26 +- 0.05} & \num{13.55 +- 0.02} & \num{13.32 +- 0.01} & - & \num{0.70 +- 0.05} & \num{0.23 +- 0.02} & - \\
            2021-04-08.691 & \num{15.6} & \num{14.33 +- 0.05} & \num{13.44 +- 0.02} & \num{13.36 +- 0.01} & - & \num{0.88 +- 0.05} & \num{0.09 +- 0.03} & - \\
            2021-04-12.724 & \num{19.5} & \num{14.22 +- 0.05} & \num{13.33 +- 0.01} & \num{13.19 +- 0.01} & - & \num{0.89 +- 0.05} & \num{0.14 +- 0.02} & - \\
            2021-04-14.702 & \num{15.6} & \num{14.21 +- 0.04} & \num{13.51 +- 0.03} & \num{13.29 +- 0.02} & - & \num{0.71 +- 0.05} & \num{0.21 +- 0.03} & - \\
            2021-04-15.714 & \num{15.6} & \num{14.23 +- 0.03} & \num{13.51 +- 0.01} & \num{13.28 +- 0.02} & - & \num{0.72 +- 0.04} & \num{0.23 +- 0.02} & - \\
            2021-05-04.736 & \num{11.0} & \num{14.32 +- 0.04} & \num{13.75 +- 0.04} & \num{13.42 +- 0.04} & \num{13.25 +- 0.03} & \num{0.57 +- 0.05} & \num{0.33 +- 0.06} & \num{0.17 +- 0.05} \\
            2021-05-10.735 & \num{15.6} & \num{13.96 +- 0.03} & \num{13.24 +- 0.01} & \num{13.05 +- 0.03} & \num{12.80 +- 0.04} & \num{0.72 +- 0.04} & \num{0.18 +- 0.03} & \num{0.26 +- 0.05} \\
            2021-05-12.724 & \num{15.6} & \num{14.14 +- 0.03} & \num{13.27 +- 0.03} & \num{13.11 +- 0.02} & \num{12.90 +- 0.03} & \num{0.87 +- 0.05} & \num{0.16 +- 0.04} & \num{0.22 +- 0.04} \\
            2021-05-14.721 & \num{14.0} & \num{14.21 +- 0.04} & \num{13.28 +- 0.04} & \num{13.13 +- 0.02} & \num{12.90 +- 0.06} & \num{0.93 +- 0.06} & \num{0.15 +- 0.04} & \num{0.24 +- 0.06} \\
            \multicolumn{9}{l}{\textbf{C/2020 P3}} \\
            2021-04-02.775 & \num{13.0} & - & \num{17.61 +- 0.05} & \num{17.24 +- 0.03} & - & - & \num{0.37 +- 0.05} & - \\
            2021-05-11.744 & \num{13.0} & - & \num{18.15 +- 0.05} & \num{17.83 +- 0.04} & - & - & \num{0.32 +- 0.06} & - \\
            2021-05-12.749 & \num{6.7} & \num{19.00 +- 0.06} & \num{18.05 +- 0.03} & \num{17.88 +- 0.02} & \num{17.67 +- 0.04} & \num{0.95 +- 0.07} & \num{0.17 +- 0.04} & \num{0.21 +- 0.05} \\
            \bottomrule
        \end{tabular}
        }
        \begin{tablenotes}
            \item[1] UT time at the beginning of exposure
            \item[2] the photometric aperture in arcsecond
        \end{tablenotes}
    \end{threeparttable}
\end{table}

% \subsection{Nucleus size}

% For comets observed at large heliocentric distances that are commonly assumed to be inactive, their photometric R magnitude, denoted as $m_R$, can be utilized to derive the maximum estimate for the geometric cross-sectional area of the cometary nucleus. This estimation involves the use of the formula proposed by \cite{lamy_comet_2004} for asteroids observed at high phase angle that can be applied to a spherical body expressed as
% \begin{equation}
%     A_R a_N^2 < \num{2.24e22} r^2 \Delta^2 10^{0.4\left(m_{\odot} - m_R + \beta \alpha\right)}, 
% \end{equation}
% where $A_R = 0.04$ is the geometric albedo \citep{lamy_comet_2004}, $a_N$ the radius, $r$ the heliocentric distance in \si{\astronomicalunit}, $\Delta$ the geocentric distance in \si{\astronomicalunit}, $m_{\odot} = -27.15$ the R magnitude of the Sun \citep{willmer_absolute_2018}, $\alpha$ the phase angle, and $\beta = 0.04$ the phase coefficient \citep{lamy_comet_2004}. 
% Given that two comets in this work were active during the observational run, we set the photometry aperture equal to stellar FWHM and calculate the background value using a circular region near the comet nucleus to reduce the influence of cometary outbursts, then use differential photometry to get new R magnitude to provide a coarse estimation for the nucleus size. 
% The images of comet C/2019 L3 taken on \DTMdate{2021-5-4} and that of C/2020 P3 taken on \DTMdate{2021-5-12} exhibit high resolution and were captured at sufficiently distant heliocentric distance, from which we were able to estimate the upper limits of radii for two comets as follows: C/2019 L3 with a limit of {\SI{75.1 +- 3.5}{\km}}, and C/2020 P3 with a limit of {\SI{26.8 +- 0.7}{\km}}. 

% % 在后面讨论一下减除彗发的方法是否可靠 (是否稳态彗发? 稳态时可靠)

\subsection{Surface brightness profile}

Based on the photometric results, the radial surface brightness profile (SBP) was computed as the function of the angular distance $\rho$ measured from the photocenter of comet. For this purpose, every image was trimmed from the center of the comet, and the sky value was determined by the median of the trimmed image. 
In the case of a steady-state coma, the surface brightness $B$ is expected to follow a power-law relation with $\rho$ as $B \propto \rho^m$, where $-1.5 \leqslant m \leqslant -1.0$, and the index $m$ is often referred to as the gradient ($m=\dif\lg{B} \big/ \dif\lg{\rho}$). As the radiation pressure accelerates the dust particles, the value of $m$ decreases and approaches \num{-1.5} in the limiting case~\citep{jewitt_surface_1987}. Conversely, if the index $m$ falls below \num{-1.5}, it suggests the presence of nonsteady dust coma emission \citep{lowry_ccd_1999}.

Not only in single images does comet C/2019 L3 appear like a stellar, but also in stacked images. However, the SBP of C/2019 L3 shows it clearly the excess flux in outer region compared with stellar SBP. In Fig.~\ref{fig:sbp} we report an example plot of the R-band SBP as a function of $\lg{\rho}$ for C/2019 L3 observed on \DTMdate{2021-5-4}. The gradient $m$ in the $0.5 \leqslant \lg{\rho} \leqslant 1.0$ range is calculated by the least-squares fit to $\lg{B}$ versus $\lg{\rho}$. 
In Fig.~\ref{fig:sbp_m}, the gradients of C/2019 L3 on each date are depicted, and the averaged value is indicated by a blue dotted line. 
Note that in Fig.~\ref{fig:sbp} the SBP is expressed as magnitude, and according to the relationship between magnitude and luminous intensity, the slope in such figure should be multiplied by \num{-0.4} to make the gradient $m$. The averaged $m$ is \num{-1.66}, and in most cases it is below \num{-1.5}, indicating that this comet's dust emission is in a nonsteady state. While for comet C/2020 P3, all images taken are so faint and it would bring about a great deal of uncertainty if we calculated its SBP. 

% 考虑放置最大、最小、中位数对应图像
\begin{figure}
    \centering
    \includegraphics[width=\columnwidth]{sbp-210504-3.pdf}
    \caption{An example figure for R-band surface brightness of comet C/2019 L3 \added{(ATLAS)}. The error plot is the SBP of comet, and the blue solid line is the SBP of a background star. The blue dotted line is a linear regression result of comet's SBP vs. $\lg{\rho}$ with $\lg{\rho}$ between 0.5 and 1.0, and the gradient $m$ related to the slope is marked on the graph with an arrow. }
    \label{fig:sbp}
\end{figure}

\begin{figure}
    \centering
    \includegraphics[width=\columnwidth]{sbp_m.pdf}
    \caption{The gradient $m$ of R-band surface brightness of comet C/2019 L3 \added{(ATLAS)} with date, blue dashed line is the averaged $m$, around which is a red shadow representing the error. }
    \label{fig:sbp_m}
\end{figure}


\subsection{$Af\rho$}

The $Af\rho$ value introduced by \citet{ahearn_comet_1984}, where $A$ represents the grain albedo, $f$ the filling factor, and $\rho$ the aperture, is commonly used to indicate the dust production activity of comets. Usually it is expressed in units of [\unit{\cm}] as equation (\ref{eq:afr}): 
\begin{equation}
    Af\rho = \frac{4 r^2 \Delta^2}{\rho} 10^{0.4(M_\odot - M_\mathrm{c})}, 
    \label{eq:afr}
\end{equation}
where $r$ is the heliocentric distance in units of [\unit{\astronomicalunit}], $\Delta$ the geocentric distance in units of [\unit{\km}], $\rho$ the aperture in units of [\unit{\km}], $M_\odot$ the absolute magnitude of the Sun (respectively \num{-26.13} and \num{-27.15} in B and R bands, see \citealt{willmer_absolute_2018}), and $M_\mathrm{c}$ the corresponding magnitude of comet under the aperture of $\rho$. 

Due to the phase darkening effect, it is necessary to adjust $Af\rho$ values at different phase angles to a specific angle. In this work, all observations were conducted at small phase angles, and we normalize the $Af\rho$ values to a phase angle of \ang{0} using equation (\ref{eq:a0frho}), as shown below:
\begin{equation}
    A(0)f\rho = \frac{A(\alpha)f\rho}{\phi(\alpha)}, \label{eq:a0frho}
\end{equation}
where $\alpha$ is the phase angle, and $\phi$ is the phase function. A composite phase function (see \citealt{schleicher_composition_2011, marcus_forward-scattering_2007}) suggested by D. Schleicher\footnote{\url{https://asteroid.lowell.edu/comet/dustphase.html}} is suitable for adjustion in this work. The related tabular data (\verb|dustphaseHM_table.txt|\footnote{\url{https://asteroid.lowell.edu/comet/dustphaseHM_table.html}}) provides the phase function with phase angle in the $\ang{0} \leqslant \alpha \leqslant \ang{180}$ range, and we adopt cubic spline interpolation method on it to obtain unlisted values. 

Fig.~\ref{fig:Afrho} shows some part of the R-band $Af\rho$ profiles, and results for the maximum of $Af\rho$ as well as the corresponding aperture on each date are summarised in Table~\ref{tab:afrho}. When a comet possesses steady coma, its $Af\rho$ will be independent of aperture. In this paper, as is shown in Fig.~\ref{fig:Afrho}, that is not the case for comet C/2019 L3 or C/2020 P3, both of which reveal a steep increse in $Af\rho$ with the aperture $\rho$ near the comet center along with a smooth decrese with larger aperture. The increase results from usually the effect of seeing and observational circumstance, while the nonsteady dust emmission and possibly the fading or destruction of dust grain bring the decrease \citep{lara_behaviour_2003,tozzi_imaging_2003}.  

\begin{figure}
    \centering
    % \subcaptionbox{}{
    %     \includegraphics[width=.6\columnwidth]{R-Afrho-C2019L3-210415.pdf}
    % }
    \subcaptionbox{R-band $A(0)f\rho$ of comet C/2019 L3 \added{(ATLAS)} on \DTMdate{2021-5-14}.}{
        \includegraphics[width=.98\columnwidth]{R-A0frho-C2019L3-210514-119.pdf}
    }

    \subcaptionbox{R-band $A(0)f\rho$ of comet C/2020 P3 \added{(ATLAS)} on \DTMdate{2021-4-2}.}{
        \includegraphics[width=.98\columnwidth]{R-A0frho-C2020P3-210402.pdf}
    }
    % \subcaptionbox{}{
    %     \includegraphics[width=.6\columnwidth]{R-Afrho-C2020P3-210512.pdf}
    % }
    \caption{R-band $A(0)f\rho$ of comet C/2019 L3 \added{(ATLAS)} and C/2020 P3 \added{(ATLAS)}. }
    \label{fig:Afrho}
\end{figure}

% 统一至20000km?
% Afrho
\begin{table}
    \centering
    \caption{$Af\rho$ values for comets C/2019 L3 \added{(ATLAS)} and C/2020 P3 \added{(ATLAS)}. }\label{tab:afrho}
    \begin{threeparttable}
        \resizebox{\linewidth}{!}{
        \begin{tabular}{cccc}
            \toprule
            Observation Time & \makecell{$Af\rho$ [\unit{\cm}] \\ ($\rho =$ \qty{e4}{\km})} & $Af\rho_\mathrm{max}$ [\unit{\cm}] & $\rho_\mathrm{max}$ [\unit{\km}]\\
            \midrule
            \multicolumn{4}{l}{\textbf{C/2019 L3}} \\
            2021-03-28.729 & \num{13611 +- 1874} & \num{13615 +- 1875} & \num{10373} \\
            2021-04-02.739 & \num{9332 +- 539} & \num{10748 +- 618} & \num{20930} \\
            2021-04-03.719 & \num{7959 +- 143} & \num{10869 +- 729} & \num{22093} \\
            2021-04-04.691 & \num{6220 +- 69} & \num{9003 +- 114} & \num{20930} \\
            2021-04-08.691 & \num{8985 +- 603} & \num{8700 +- 110} & \num{19860} \\
            2021-04-12.724 & \num{7911 +- 102} & \num{7410 +- 269} & \num{29343} \\
            2021-04-14.702 & \num{7320 +- 95} & \num{8268 +- 133} & \num{25822} \\
            2021-04-15.714 & \num{5043 +- 244} & \num{8891 +- 154} & \num{19953} \\
            2021-05-04.736 & \num{5943 +- 96} & \num{8320 +- 340} & \num{29365} \\
            2021-05-10.735 & \num{7523 +- 134} & \num{8787 +- 205} & \num{29343} \\
            2021-05-12.724 & \num{5701 +- 236} & \num{9590 +- 157} & \num{21127} \\
            2021-05-14.721 & \num{6356 +- 152} & \num{9164 +- 95} & \num{29343} \\
            \multicolumn{4}{l}{\textbf{C/2020 P3}} \\
            2021-04-02.775 & \num{869 +- 20} & \num{948 +- 19} & \num{15789} \\
            2021-05-11.744 & \num{708 +- 17} & \num{848 +- 17} & \num{19048} \\
            2021-05-12.749 & \num{606 +- 31} & \num{801 +- 38} & \num{19048} \\
            \bottomrule
        \end{tabular}
        }
    \end{threeparttable}
\end{table}

\subsection{Coma Colors}

The photometric results in Table~\ref{tab:bvr} also summarises the color index of the two comets. 
Fig.~\ref{fig:color-r} is the color indices $\mathrm{B-V}$, $\mathrm{V-R}$, and $\mathrm{R-I}$ of C/2019 L3 as a function of heliocentric distance in units of [\unit{\astronomicalunit}], all exhibiting variability during the approach to perihelion. 
As C/2019 L3 approached perihelion, its $\mathrm{V-R}$ color index showed a tendency towards blue. 
Considering its position at a heliocentric distance of about {\qty{4}{\astronomicalunit}}, with a temperature of around {\qty{140}{\K}}, the gas-driven effects are significant. 
The colors for C/2019 L3 are as follows: 
$\langle \mathrm{B-V} \rangle = \num{0.75 +- 0.06}$, 
$\langle \mathrm{V-R} \rangle = \num{0.27 +- 0.05}$, and 
$\langle \mathrm{R-I} \rangle = \num{0.22 +- 0.05}$. 
Besides, the colors for C/2020 P3 are 
$\mathrm{B-V} = \num{0.95 +- 0.07}$, 
$\langle \mathrm{V-R} \rangle = \num{0.29 +- 0.05}$, and 
$\mathrm{R-I}= \num{0.21 +- 0.05}$. 

\begin{figure}
    \centering
    \includegraphics[width=\linewidth]{color-r.pdf} 
    \caption{Color indices vs.\ heliocentric distance plot for comet C/2019 L3 \added{(ATLAS)}.}\label{fig:color-r}
\end{figure}

Moreover, in order to indicate how the scattered color of the dusty coma varies with wavelengths, the reddening $\mathcal{R}$ (or normalized color) \citep{jewitt_cometary_1986, lara_behaviour_2003, mazzotta_epifani_dust_2011, shi_ccd_2015} is calculated in units of [\unit{\percent/\kilo\angstrom}], with the formula given as equation (\ref{eq:red}): 
\begin{equation}
\mathcal{R} = \frac{2}{Af\rho_1 + Af\rho_2} \frac{Af\rho_2 - Af\rho_1}{\lambda_2 - \lambda_1}, 
\label{eq:red}
\end{equation}
where $\lambda_1$ and $\lambda_2$ are the central wavelengths of the filters in units of [\unit{\nm}] (respectively the centers of B and R filters, {\qty{440}{\nm}} and {\qty{658}{\nm}}). With this parameter it is convenient to indicate the percentage of change in the strength of the continuum per {\qty{1000}{\angstrom}}. The results of dust reddening are summarised in Table~\ref{tab:reddening}. In order to avoid the possible effects from background residuals, it is calculated with aperture of {\qty{2e4}{\km}}. 

For comet C/2019 L3, the reddening undergoes significant variations over time, from positive value {\qty{13.75 +- 1.07}{\percent/\kilo\angstrom}} on \DTMdate{2021-3-28} to negative value {\qty{-15.69 +- 0.37}{\percent/\kilo\angstrom}} on \DTMdate{2021-4-12}. We plot the reddening as a function of date in Fig.~\ref{fig:reddening}, and the averaged value {\qty{0.94 +- 0.23}{\percent/\kilo\angstrom}} is marked as a horizontal blue dotted line. For comet C/2020 P3, only on \DTMdate{2021-5-12} is its observational data sufficient for the calcltaion of reddening with a result of {\qty{-6.65 +- 0.01}{\percent/\kilo\angstrom}}. 

The reddening of C/2019 L3 on each observation date is shown in Fig.~\ref{fig:reddening}, with the blue dotted line indicating the mean value. The plot reveals that over the observation course of nearly two months, C/2019 L3 exhibited multiple transitions from a reddish to a bluish color, possibly due to variations in the composition of the coma \citep{ivanova_colour_2017}. 

\begin{table}
    \centering
    \caption{Reddening of comets C/2019 L3 \added{(ATLAS)} and C/2020 P3 \added{(ATLAS)} with $\rho = $ \qty{2e4}{\km}. }\label{tab:reddening}
    \begin{threeparttable}
        \resizebox{\linewidth}{!}{
        \begin{tabular}{cccc}
            \toprule
            Observation Time & B-band $Af\rho$ & R-band $Af\rho$ & reddening [\unit{\percent/\kilo\angstrom}]\\
            \midrule
            \multicolumn{4}{l}{\textbf{C/2019 L3}} \\
            2021-03-28.729 & \num{9472 +- 753} & \num{12811 +- 1764} & \num{13.75 +- 1.07}\\
            2021-04-02.739 & \num{8763 +- 806} & \num{10719 +- 617} & \num{9.21 +- 0.50}\\
            2021-04-03.719 & \num{9886 +- 274} & \num{9538 +- 156} & \num{-1.65 +- 0.03}\\
            2021-04-04.691 & \num{9501 +- 343} & \num{8734 +- 93} & \num{-3.86 +- 0.07}\\
            2021-04-08.691 & \num{9288 +- 402} & \num{10797 +- 725} & \num{6.89 +- 0.27}\\
            2021-04-12.724 & \num{12683 +- 602} & \num{8979 +- 113} & \num{-15.69 +- 0.37}\\
            2021-04-14.702 & \num{8809 +- 382} & \num{8700 +- 110} & \num{-0.57 +- 0.01}\\
            2021-04-15.714 & \num{6077 +- 283} & \num{7173 +- 271} & \num{7.59 +- 0.23}\\
            2021-05-04.736 & \num{7724 +- 264} & \num{8065 +- 130} & \num{1.98 +- 0.04}\\
            2021-05-10.735 & \num{9030 +- 285} & \num{8891 +- 154} & \num{-0.71 +- 0.01}\\
            2021-05-12.724 & \num{9044 +- 336} & \num{7958 +- 326} & \num{-5.86 +- 0.16}\\
            2021-05-14.721 & \num{8428 +- 258} & \num{8475 +- 199} & \num{0.25 +- 0.01}\\
            \multicolumn{4}{l}{\textbf{C/2020 P3}} \\
            2021-05-12.749 & \num{934 +- 74} & \num{799 +- 37} & \num{-6.65 +- 0.01}\\
            \bottomrule
        \end{tabular}
        }
    \end{threeparttable}
\end{table}

\begin{figure}
    \centering
    \includegraphics[width=\columnwidth]{reddening.pdf}
    \caption{The reddening of comet C/2019 L3 \added{(ATLAS)} with date, blue dashed line is the averaged reddening, and the surounding red shadow shows the error. }\label{fig:reddening}
\end{figure}


\section{Discussion}\label{sec:dis}


% From \autoref{tab:afrho} we can conclude that the comet C/2019 L3 is very active due to its high $A(0)f\rho$ values during the observational run, while C/2020 P3 is less active since its $A(0)f\rho$ values are only about {\SI{10}{\percent}} of that of C/2019 L3. 
Based on the data presented in Table~\ref{tab:afrho}, it can be inferred that the comet C/2019 L3 exhibited a high level of activity during the observation period, as indicated by its significantly higher $A(0)f\rho$ values compared to C/2020 P3. In contrast, the $A(0)f\rho$ values of C/2020 P3 were only about \SI{10}{\percent} of those of C/2019 L3. 
Comparing $A(0)f\rho$ values with that of other LPCs, just as Fig.~\ref{fig:afrho-ref} shown, C/2019 L3 is very active at heliocentric distance of $\thicksim${\SI{4}{\astronomicalunit}}, and C/2020 P3 is moderately active at heliocentric distance of $\thicksim${\SI{7}{\astronomicalunit}}. 
\st{Moreover, the R-band $A(0)f\rho$ values in the reference aperture of {\SI{e4}{\km}} appear a slight decrese in trend as it approached to perihelion viewed from Fig.~\ref{fig:a0frho-c2019}, so it seems that the heat of the Sun plays a minor role on the activity of C/2019 L3 during the observational period.} \ul{Moreover, the $A(0)f\rho$ values of C/2019 L3 in the R-band for the reference aperture of {\SI{e4}{\km}} exhibit a slight decreasing trend as the comet approached perihelion, as shown in Fig.~\ref{fig:a0frho-c2019}. This suggests an abnormal activity of the comet. }% 在此只写活动性显现出异常, 不作过度猜测
On the other hand, 
the BC-band $A(0)f\rho$ of C/2019 L3 up to \SI{24710 +- 125}{\cm} on \DTMdate{2022-1-19} posted on The Astronomer's Telegram\footnote{\href{https://www.astronomerstelegram.org/?read=15186}{https://www.astronomerstelegram.org/?read=15186}}, with heliocentric distance $r = \SI{3.56}{\astronomicalunit}$ and geocentric distance $\Delta = \SI{2.61}{\astronomicalunit}$, indicates that C/2019 L3 appears very active near the perihelion. 


% 与其他LPC比较,相位角均校正至0
\begin{figure}
    \centering
    \includegraphics[width=\columnwidth]{a0frho-r.pdf}
    \caption{$A(0)f\rho$ values measured in this work compared with that of other LPCs in  literature by \citet{mazzotta_epifani_observational_2014}, \citet{garcia_photometry_2021}, \citet{garcia_observational_2020}, \citet{rousselot_monitoring_2014}, \citet{meech_activity_2009}, \citet{sarneczky_activity_2016}, \citet{solontoi_ensemble_2012}, \citet{szabo_spectrophotometry_2002}. All data have been adjusted to phase angle \ang{0}. }\label{fig:afrho-ref}
\end{figure}% data amount error

\begin{figure}
    \centering
    \includegraphics[width=\columnwidth]{a0frho-r-C2019L3.pdf}
    \caption{R-band $A(0)f\rho$ values of C/2019 L3 as a function of heliocentric distance. }\label{fig:a0frho-c2019}
\end{figure}

According to the study by \cite{ramirez_ubvric_2012}, the solar colors are as follows: 
${(\mathrm{B} - \mathrm{V})}_{\odot} = \num{0.653 +- 0.005}$, 
${(\mathrm{V} - \mathrm{R})}_{\odot} = \num{0.352 +- 0.007}$, and 
${(\mathrm{R} - \mathrm{I})}_{\odot} = \num{0.350 +- 0.009}$. 
The average color indices of two Long-period comets and three dynamically new comets were calculated by \cite{meech_activity_2009} to be 
$\langle \mathrm{B} - \mathrm{V} \rangle = \num{0.76 +- 0.01}$ and 
$\langle \mathrm{V} - \mathrm{R} \rangle = \num{0.43 +- 0.01}$, 
with their heliocentric distances ranging from \SIrange{5.8}{14.0}{\astronomicalunit}. 
\cite{solontoi_ensemble_2012} studied six Long-period comets within \SI{5}{\astronomicalunit} of the Sun and obtained the average color indices of 
$\langle \mathrm{B} - \mathrm{V} \rangle = \num{0.687 +- 0.005}$ and 
$\langle \mathrm{V} - \mathrm{R} \rangle = \num{0.443 +- 0.003}$. 
\cite{jewittCOLORSYSTEMATICSCOMETS2015} investigated Long-period comets with large range of heliocentric distances (\SIrange{1.875}{17.982}{\astronomicalunit}), and the average color indices were found to be 
$\langle \mathrm{B} - \mathrm{V} \rangle = \num{0.78 +- 0.02}$, 
$\langle \mathrm{V} - \mathrm{R} \rangle = \num{0.47 +- 0.02}$, and 
$\langle \mathrm{R} - \mathrm{I} \rangle = \num{0.42 +- 0.03}$. 

Fig.~\ref{fig:color_index}a is the color indices $\mathrm{B}-\mathrm{V}$, $\mathrm{V}-\mathrm{R}$, and $\mathrm{R}-\mathrm{I}$ of C/2019 L3 as a function of heliocentric distance in \si{\astronomicalunit}, all exhibiting variability during the approach to perihelion. 
As C/2019 L3 approached perihelion, its $\mathrm{V}-\mathrm{R}$ color index showed a tendency towards blue. 
Considering its position at a heliocentric distance of about {\SI{4}{\astronomicalunit}}, with a temperature of around {\SI{140}{\K}}, the gas-driven effects are significant. 

Fig.~\ref{fig:color_index}b is the $\langle \mathrm{B}-\mathrm{V} \rangle$ versus $\langle \mathrm{V}-\mathrm{R} \rangle$ plot of C/2019 L3, C/2020 P3 and other LPCs. 
The color index of the Sun \citep{ramirez_ubvric_2012} is marked as a red circle with dot. 
As we can see, the $\langle \mathrm{B} - \mathrm{V} \rangle$ colors of two comets are redder than the Sun, while the $\langle \mathrm{V} - \mathrm{R} \rangle$ colors of them are bluer than the Sun. 
Compared with LPCs in other works, the $\langle \mathrm{B} - \mathrm{V} \rangle$ color of C/2019 L3 is consistent with them, while the $\langle \mathrm{V} - \mathrm{R} \rangle$ color of C/2019 L3 is significantly bluer than them. For C/2020 P3, the $\langle \mathrm{B} - \mathrm{V} \rangle$ color is redder while the $\langle \mathrm{V} - \mathrm{R} \rangle$ color is bluer. 
In general, the color indices of the two LPCs studied in this work differ from those of other LPCs. 

The reddening of C/2019 L3 on each observation date is shown in Fig.~\ref{fig:reddening}, with the blue dotted line indicating the mean value. The plot reveals that over the observation course of nearly two months, C/2019 L3 exhibited multiple transitions from a reddish to a bluish color, possibly due to variations in the composition of the coma \citep{ivanova_colour_2017}.

% V-R随日心距呈现变蓝的趋势, 在4au左右CO2的升华作用较明显, 因此, C/2019 L3的活动性可能是由于CO2的升华导致

\begin{figure}
    \centering
    \subcaptionbox{Color indices vs.\ heliocentric distance for C/2019 L3}{
        \includegraphics[width=\linewidth]{color-r.pdf} 
    }
    \subcaptionbox{Color indices $\langle \mathrm{B} - \mathrm{V} \rangle$ vs. $\langle \mathrm{V} - \mathrm{R} \rangle$ plot of Long-period comets}{
        \includegraphics[width=\linewidth]{color-color.pdf}
    }
   
    \caption{Color index plot \ul{of C/2019 L3, C/2020 P3 and other Long-period comets.} }\label{fig:color_index}
\end{figure}




\section{Conclusions} \label{sec:con}
In this work, we present the observational results for two Long-period comets, namely C/2019 L3 and C/2020 P3. 

An analysis of morphology was conducted on combined comet images with several enhancement techniques. Unfortunately, most of them show no obvious features, except for image of C/2019 L3 on \DTMdate{2021-5-14}, where there exists a small northeastward fan-shape structure under V and R filters. 

Circular aperture photometry was employed to measure the magnitudes of comets in various broadband filters within an aperture size of up to 1.5 times the full-width at half-maximum (FWHM) of the comet. 

We get the surface brightness profile for comet C/2019 L3 and calculate the gradient in the $0.5 \leqslant \lg{\rho} \leqslant 1.0$ range, finding that the average value is \num{-1.68}, suggesting a nonsteady coma. 

The $Af\rho$ values adjusted to phase angle of \ang{0} were derived based on photometric results in the aperture of \SI{e4}{\km}. Compared to other works, the $A(0)f\rho$ of C/2019 L3 is relatively high at $\thicksim${\SI{4}{\astronomicalunit}}, while that of C/2020 P3 is moderate at $\thicksim${\SI{7}{\astronomicalunit}}. The profiles of $Af\rho$ also show that both of the two comets possess a nonsteaty coma, since the $Af\rho$ values tend to decrease with aperture. 

Coma colors were described in the form of color indices and reddening. The $\mathrm{B} - \mathrm{V}$ colors of C/2019 L3 and C/2020 P3 are redder than the Sun, while the $\mathrm{V} - \mathrm{R}$ and $\mathrm{R} - \mathrm{I}$ colors of them are bluer than the Sun. The reddening of C/2019 L3 exhibits variations during the observational runs. 




\section*{Acknowledgements}

The Acknowledgements section is not numbered. Here you can thank helpful
colleagues, acknowledge funding agencies, telescopes and facilities used etc.
Try to keep it short.

%%%%%%%%%%%%%%%%%%%%%%%%%%%%%%%%%%%%%%%%%%%%%%%%%%
\section*{Data Availability}

 
The inclusion of a Data Availability Statement is a requirement for articles published in MNRAS. Data Availability Statements provide a standardised format for readers to understand the availability of data underlying the research results described in the article. The statement may refer to original data generated in the course of the study or to third-party data analysed in the article. The statement should describe and provide means of access, where possible, by linking to the data or providing the required accession numbers for the relevant databases or DOIs.




%%%%%%%%%%%%%%%%%%%% REFERENCES %%%%%%%%%%%%%%%%%%

% The best way to enter references is to use BibTeX:

\bibliographystyle{mnras}
\bibliography{books} 


% Alternatively you could enter them by hand, like this:
% This method is tedious and prone to error if you have lots of references
%\begin{thebibliography}{99}
%\bibitem[\protect\citeauthoryear{Author}{2012}]{Author2012}
%Author A.~N., 2013, Journal of Improbable Astronomy, 1, 1
%\bibitem[\protect\citeauthoryear{Others}{2013}]{Others2013}
%Others S., 2012, Journal of Interesting Stuff, 17, 198
%\end{thebibliography}

%%%%%%%%%%%%%%%%%%%%%%%%%%%%%%%%%%%%%%%%%%%%%%%%%%

%%%%%%%%%%%%%%%%% APPENDICES %%%%%%%%%%%%%%%%%%%%%

\appendix

\section{Some extra material}

If you want to present additional material which would interrupt the flow of the main paper,
it can be placed in an Appendix which appears after the list of references.

%%%%%%%%%%%%%%%%%%%%%%%%%%%%%%%%%%%%%%%%%%%%%%%%%%


% Don't change these lines
\bsp	% typesetting comment
\label{lastpage}
\end{document}

% End of mnras_template.tex
